\documentclass[11pt]{article}
\usepackage{graphicx} % Required for inserting images
\usepackage{titling}
\usepackage{geometry}
\geometry{a4paper, margin=1in}
\usepackage[utf8]{inputenc}
\usepackage{amsmath, amssymb, mathtools}
\usepackage{physics, braket, slashed}
\numberwithin{equation}{section}
\AtBeginDocument{\Large\selectfont}
\title{Notes for Path Integral Quantization}
\author{YILUN LIN}
\date{June 2025}

\begin{document}

\maketitle

\section{Principle of Path Integral}
\textbf{Canonical Quantization vs. Path Integral Quantization:}
\begin{enumerate}
    \item Canonical quantization relies on commutation relations and operator
    formalism.
    \item Path integral quantization achieves second quantization by expressing quantum amplitudes as a sum over histories.
\end{enumerate}
\textbf{Basic Idea of Path Integral:}\\
For a scalar particle (i.e. a $spin=0$ particle) in quantum mechanics:
\begin{equation}
    \langle x_f,t_f\mid x_i,t_i\rangle\,=\,\int\!\mathcal{D}x(t)e^{i\mathcal S[x]/\hbar},
\end{equation}
where $\mathcal Dx(t)$ denotes all possible paths and $\mathcal S[x]$ is just the classical action. Actually, this is exactly the core difference between quantum mechanics and quantum field theory. Quantum mechanics can describe motion of one-particle states in the non-relativistic limit by using wavefunction $\lvert\psi\rangle$ and only the classical action. While when it comes to multi-particle states in relativistic situation, in order to cancel the effects of disconnected diagrams arising in the process of second quantization, we need to take the logarithm of the path integral and derive the effective action using the Legendre transform. And this is what people always do in the quantum field theory formalism.\\[0.5em]
\textbf{From Mechanics to Field Theory:}\\
The involvement of special relativity into quantum theory implies that particles in the space can be created and annihilated since one-particle states could not exist steadily. While traditional quantum mechanics does not include such creation and annihilation of particles, the application of field theory method is essential to establishing a proper theoretical framework.\\
To transform from traditional quantum mechanics to quantum field theory, coordinates $x(t)$ need to be promoted to fields $\phi(\Vec{x})$, and the sum over paths $\int\!\mathcal Dx(t)$ becomes a functional integral $\int\!\mathcal D\phi(\Vec{x})$.\\
The central concept in path integral quantization is the generating functional:
\begin{equation}
    \mathcal Z[J]\,=\,\int\!\mathcal D\phi\,e^{i\mathcal S[\phi]+i\int\!d^4x\,J(\Vec{x})\phi(\Vec{x})},
\end{equation}
where $\mathcal{S}[\phi]$ is the action of the field $\phi(\Vec{x})$ and $J(\Vec{x})$ is an external source.\\
Functional derivatives of $\mathcal{Z}[J]$ yield time-ordered correlation functions:
\begin{equation}
    \langle0\mid\mathcal T\{\phi(\Vec{x_1}),\cdots,\phi(\Vec{x_n})\}\mid0\rangle\,=\,\left.\frac{1}{\mathcal{Z}[0]}\,\frac{\delta^n\mathcal{Z}[J]}{\delta J(\Vec{x_1})\cdots\delta J(\Vec{x_n})}\right|_{J=0}
\end{equation}
For convergence, we often perform a Wick rotation $t\rightarrow -i\tau$ to yield a Euclidean path integral:
\begin{equation}
    \mathcal{Z}[J]\,=\,\int\!\mathcal{D}\phi\,e^{-\mathcal{S}_{E}[\phi]+\int\!d^4x\,J(\Vec{x})\phi(\Vec{x})}.
\end{equation}

\section{Path Integral Quantization of Klein-Gordon Fields}
The Lagrangian for a real scalar field is
\begin{equation}
    \mathcal{L}\,=\,\frac{1}{2}\partial_\mu\phi\,\partial^\mu\phi-\frac{1}{2}m^2\phi^2,
\end{equation}
thus the action is
\begin{equation}
    \mathcal{S}[\phi]\,=\,\int\!d^4x[\frac{1}{2}(\partial_\mu\phi)^2-\frac{1}{2}m^2\phi^2].
\end{equation}
The generating functional with source $J(\Vec{x})$ is
\begin{equation}
    \mathcal{Z}[J]\,=\,\int\!\mathcal{D}\phi\,e^{i\int\!d^4x\,\left[\frac{1}{2}\phi(\Box+m^2)\phi\,+\,J(\Vec{x})\phi\right]},
\end{equation}
which can be computed exactly since it is a Gaussian functional integral.
In order to solve this Gaussian integral, we consider to complete the square:
\begin{equation}
    \phi(\Vec{x})\,=\,\phi_c(\Vec{x})\,+\,\phi_q(\Vec{x}),
\end{equation}
where
\begin{equation}
    \phi_c(\Vec{x})\,=\,-\int\!d^4x'\,\Delta_F(\Vec{x}-\Vec{x'})J(\Vec{x'}).
\end{equation}
Therefore,
\begin{equation}
    \mathcal{Z}[J]\,=\,\mathcal{Z}[0]\,e^{-\frac{i}{2}\int\!d^4x\,d^4x'J(\Vec{x})\Delta_F(\Vec{x}-\Vec{x'})J(\Vec{x'})}.
\end{equation}
Here $\Delta_F(\Vec{x}-\Vec{x'})$ is the Feynman propagator which satisfies
\begin{equation}
    (\Box\,+m^2)\,\Delta_F(\Vec{x}-\Vec{x'})\,=\,-i\delta^4(\Vec{x}-\Vec{x'}).
\end{equation}

\section{Path Integral Quantization of Dirac Field}
The Lagrangian for a free spinor field is
\begin{equation}
    \mathcal{L}\,=\,\bar\psi(i\gamma^\mu\partial_\mu-m)\psi
\end{equation}
and the action is
\begin{equation}
    \mathcal{S}[\bar\psi,\psi]\,=\,\int\!d^4x\,\bar\psi(i\gamma^\mu\partial_\mu-m)\psi.
\end{equation}
The Dirac spinor field is used to describe particles with a spin $\frac{1}{2}$, which are kinds of Fermions. While Fermionic fields are quantized using Grassmann variables, which obey the antisymmetry:
\begin{equation}
    \psi_i\psi_j\,=\,-\psi_j\psi_i.
\end{equation}
Integration over Grassmann variables is defined by
\begin{equation}
    \begin{split}
        \int\!d\psi\,&=\,0,\\ \int\!d\psi\,\psi\,&=\,1.
    \end{split}
\end{equation}
The generating functional with Grassmann sources $\eta$ and $\bar\eta$ is
\begin{equation}
    \mathcal{Z}[\bar\eta,\eta]\,=\,\int\!\mathcal{D}\bar\psi\,\mathcal{D}\psi\,e^{i\int\!d^4x\,[\bar\psi(i\gamma^\mu\partial_\mu-m)\psi\,+\,\bar\eta\psi\,+\,\bar\psi\eta]}.
\end{equation}
For a bilinear fermionic action, we have:
\begin{equation}
    \mathcal{Z}[\bar\eta,\eta]\,=\,\det(i\slashed{\partial}-m)\,e^{-i\int\!d^4x\,d^4x'\bar\eta(\Vec{x})\,S_F(\Vec{x}-\Vec{x'})\,\eta(\Vec{x'})},
\end{equation}
where $S_F(\Vec{x}-\Vec{x'})$ is the Feynman Propagator for the Dirac Field, satisfying
\begin{equation}
    (i\slashed{\partial}_x\,-\,m)\,S_F(\Vec{x}-\Vec{x'})\,=\,\delta^4(\Vec{x}-\Vec{x'}).
\end{equation}



\end{document}
